\newcommand{\basedir}{fablab-document/}
\documentclass{\basedir/fablab-document}

\title{Richtlinien zur Benennung von Lagerplätzen}
\author{Michael Jäger, Patrick Kanzler}

\fancyfoot[L]{https://github.com/fau-fablab/docs}
\fancyfoot[C]{}
%\fancyfoot[R]{Liste Nr. \underline{~~~~~~~~}}

\begin{document}
%optional: dicke Überschrift auf der ersten Seite
\vspace*{-7em}\maketitle

Die Lagerplätze im FabLab sollen einheitlich kodiert sein, um Dinge schneller zu finden.
Der Lagerort soll sowohl für neue, als auch für alte Artikel nach und nach angepasst werden.
Zusätzlich müssen alle häufig verwendeten Lagerorte noch menschenlesbar beschriftet werden.
(Bei den Elektronikkisten z.\,B. E2 {\it Buchsen und Stecker})

\textbf{Beispiele}

\textit{Feilen - Werkbank, Schublade Nr. 2 vorne links}\newline
W2-VL

\textit{Rundmaterial 50mm Stahl - Drehbank, Schublade, Hinten}\newline
D1-H

\textbf{Die Lagerkennung ist wie folgt zu bilden:}

\begin{enumerate}
\item Ort (im ERP als Lagerort auswählen, beim einzelnen Produkt oder für dessen Kategorie)
\begin{enumerate}
\item Grober Lagerort (Buchstabe)
	
	\begin{tabular}{c|c|c|c}
	\hline
	K & \textbf{K}ellerlager & W & \textbf{W}erkbank \\
	\hline
	E & \textbf{E}lektronikregal & D & \textbf{D}rehbank \\
	\hline
	C & \textbf{C}hemietisch & A & \textbf{A}crylregal (inkl. 3D-Drucker und Plotter) \\
	\hline
	S & \textbf{S}chraubenregal & V & \textbf{V}itrine \\
	\hline
	B & Schrank im \textbf{B}esprechungsraum & F & \textbf{F}räserkasten \\
	\hline
	L & \textbf{L}aborschubladen an Elektronikarbeitsplätzen & & \\
	\hline
	\end{tabular}
\item Genauerer Lagerort (Zahl) \newline
	z.\,B. eine Schubladen- oder Kistennummer\newline
    Durchnummeriert 1, 2, 3, ...\newline
Oder mit Zeile und Spalte 1.1 (links oben) ... 1.99 (rechts oben), 2.1, ...
\end{enumerate}
\textcolor{gray}{
\item    Bindestrich --
\item   optional: Genauer Platz (Zahl oder Buchstabe) (im ERP mit den Freitext-Feldern, die verwirrender Weise \enquote{Regal, Zeile, Vorfall} heißen)
	\begin{itemize}
	\item Fachnummer \newline
	1, 2, 3, ...
	\item Position\newline
    V - Vorne\newline
    H - Hinten\newline
    L - links\newline
    M - Mitte\newline
    R - Rechts \newline
    oder kombiniert z.B. HR - hinten rechts\newline
	\end{itemize}
}
\end{enumerate}
\end{document}
