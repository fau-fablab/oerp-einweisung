%%%%%%%%%%%%%%%%%%%%%%%%%%%%%%%%%%%%%%%%%%%%%%%%
% COPYRIGHT: (C) 2012-2015 FAU FabLab and others
% Bearbeitungen ab 2015-02-20 fallen unter CC-BY-SA 3.0
% Sobald alle Mitautoren zugestimmt haben, steht die komplette Datei unter CC-BY-SA 3.0. Bis dahin ist der Lizenzstatus aller alten Bestandteile ungeklärt.
%%%%%%%%%%%%%%%%%%%%%%%%%%%%%%%%%%%%%%%%%%%%%%%%


\newcommand{\basedir}{fablab-document}
\documentclass{\basedir/fablab-document}

% \usepackage{fancybox} %ovale Boxen für Knöpfe - nicht mehr benötigt
\usepackage{amssymb} % Symbole für Knöpfe
\usepackage{subfigure,caption}
\usepackage{eurosym}
\usepackage{tabularx} % Tabellen mit bestimmtem Breitenverhältnis der Spalten
\usepackage{wrapfig} % Textumlauf um Bilder

\renewcommand{\texteuro}{\euro}

\definecolor{dunkelgruen}{rgb}{0,0.4,0}

\newcommand{\img}[2]{
	\begin{figure}[ht]%
	\includegraphics[width=\columnwidth]{img/#1}%
	\caption{#2}%
	\end{figure}}
	
\newcommand\circlearound[1]{%
  \tikz[baseline]\node[draw,shape=circle,anchor=base] {#1} ;}
	
\newcounter{ueberpunkte}
\setcounter{ueberpunkte}{1}
\renewcommand{\labelenumi}{\circlearound{\theueberpunkte}\stepcounter{ueberpunkte}}

\linespread{1.2}

\date{2014}
\author{Michael Jäger}
\fancyfoot[C]{kontakt@fablab.fau.de}
\title{Produkte anlegen im ERP}

\begin{document}

\section{Anleitung}

\img{01_Kategorie_heraussuchen.pdf}{Kategorie heraussuchen in die das Produkt passt}
\begin{enumerate}
	\item Kategorieansicht auswählen
	\item Unterkategorie auswählen oder neu anlegen, falls noch nicht vorhanden
\end{enumerate}

\newpage
\img{02_Produkt_anlegen.pdf}{Produkt anlegen}
\begin{enumerate}
	\item Neues, leeres Produkt anlegen
\end{enumerate}

\newpage
\img{03_Produkt_Informationen.pdf}{Produktinformationen ausfüllen - Reiter Übersicht}

\begin{tabular}{>{\itshape}l@{\qquad}l}
\multicolumn{2}{l}{\color{red}\textbf{Müssen ausgefüllt werden}} \\
Produktbezeichnung & Sinnvolle Beschreibung durch die man den Artikel auch durch Suche findet \\
Interne Referenz & Artikelnummer im Kassensystem, 4 stellige Zahl \textbf{mit führenden Nullen!} \\
                 & (probiere eine zufällige Zahl rund um 4000) \\
                 & Wenn die Nummer bereits belegt ist, kommt beim Speichern der Fehler:\\
                 & \enquote{code of product must be unique!} \\
                 & \url{https://github.com/fau-fablab/oerp-nextid} kann auch helfen \\
Verkaufspreis & meist mindestens 1,5-facher Einkaufspreis, mindestens einige Cent mehr\\
& \\
\multicolumn{2}{l}{\color{orange}\textbf{Müssen überprüft werden}} \\
Kategorie & Sinnvolle Einsortierung um den Artikel wiederzufinden\\
Kann verkauft werden & Anhaken wenn der Artikel zum Verkauf steht\\
Produktart & \begin{tabular}{@{}>{\itshape}l@{\qquad}l} Verbrauchsmaterial & normale Verkaufsware \\
 Dienstleistung & immaterialle Güter z.\,B.\  Laserzeit \\
 Lagerprodukt & nur nach extra Bestellung (benutzen wir eigtl.\  net)
\end{tabular} \\
ME (Mengeneinheit) & Verkaufeinheit z.\,B.\  Stück, Zentimeter, Gramm \\
& \\
\multicolumn{2}{l}{\color{dunkelgruen}\textbf{Können ausgefüllt werden}} \\
EAN13 Barcode & EAN Barcode, falls bekannt \\
& \\
\end{tabular}
\begin{enumerate}
	\item Reiter Beschaffung
\end{enumerate}

\newpage
\img{04_Produkt_Beschaffung.pdf}{Produktbeschaffung ausfüllen - Reiter Beschaffung}
\begin{tabular}{>{\itshape}l@{\qquad}l}
\multicolumn{2}{l}{\color{red}\textbf{Müssen ausgefüllt werden}} \\
Herstellungskosten & Einkaufspreis \textbf{pro Verkaufseinheit}, dieser wird auch \\
                   & für Eingangsrechnungen verwendet \\
& \\
\multicolumn{2}{l}{\color{orange}\textbf{Müssen überprüpft werden}} \\
Einkauf ME (Mengeneinheit) & Wird der Artikel im z.\,B.\  100er Pack gekauft oder einzeln?\\
Hersteller & Falls bekannt bitte eintragen, vor allem bei Elektronik\\
Hersteller-Produktname & Falls bekannt bitte eintragen, vor allem bei Elektronik\\
Hersteller-Artikelnummer & Falls bekannt bitte eintragen, vor allem bei Elektronik\\
\end{tabular}
\begin{enumerate}
	\item Lieferant(en) hinzufügen
\end{enumerate}

\newpage
\img{05_Produkt_Lieferant.pdf}{Lieferant eintragen - PopUp Lieferant}
\begin{tabular}{>{\itshape}l@{\qquad}l}
\multicolumn{2}{l}{\color{red}\textbf{Müssen ausgefüllt werden}} \\
Lieferant & Lieferant auswählen oder neu anlegen, falls noch nicht vorhanden\\
Produktbezeichnung bei Lieferant & Artikel\textbf{name} beim Lieferanten\\
Kurzbezeichnung bei Partner & Bestell\textbf{nummer} beim Lieferanten \\
Minimale Menge & Mindestbestellmenge bei diesem Lieferanten \\
Tabelle  & \begin{tabular}{@{}>{\itshape}l@{\qquad}l} Menge & Einkaufsmenge beim Lieferanten z.\,B.\  Staffelpreise \\
 Stückpreis & zugehöriger Stückpreis \textbf{pro Einkaufseinheit} \\
\end{tabular} \\
& \\
\multicolumn{2}{l}{\color{dunkelgruen}\textbf{Können ausgefüllt werden}} \\
Speichern \& Neu & Zweiten Lieferanten anlegen, falls vorhanden \\
& \\
\end{tabular}
\begin{enumerate}
	\item Speichern \& Beenden
	\item Reiter Bestand
\end{enumerate}

\newpage
\img{06_Produkt_Bestand.pdf}{Lagerort eintragen - Reiter Bestand}
\begin{tabular}{>{\itshape}l@{\qquad}l}
\multicolumn{2}{l}{\color{orange}\textbf{Müssen überprüpft werden}} \\
Lagerort & Falls bekannt eintragen, ansonsten wird der Lagerort der Kategorie verwendet \\
& \\
\end{tabular}
\begin{enumerate}
	\item Reiter Finanzen
\end{enumerate}

\newpage
\img{07_Produkt_Finanzen.pdf}{Steuer eintragen - Reiter Finanzen}
\begin{tabular}{>{\itshape}l@{\qquad}l}
\multicolumn{2}{l}{\color{red}\textbf{Müssen ausgefüllt werden}} \\
Steuer des Einkaufs (Vorsteuer) & Immer inkl. 19\% MwSt \\
& \\
\end{tabular}
\begin{enumerate}
	\item Speichern \& fertig
\end{enumerate}


\newpage
\ccLicense{oerp-einweisung}{Einweisung Oerp}

\end{document}
