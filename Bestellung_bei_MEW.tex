%%%%%%%%%%%%%%%%%%%%%%%%%%%%%%%%%%%%%%%%%%%%%%%%
% COPYRIGHT: (C) 2012-2015 FAU FabLab and others
% Bearbeitungen ab 2015-02-20 fallen unter CC-BY-SA 3.0
% Sobald alle Mitautoren zugestimmt haben, steht die komplette Datei unter CC-BY-SA 3.0. Bis dahin ist der Lizenzstatus aller alten Bestandteile ungeklärt.
%%%%%%%%%%%%%%%%%%%%%%%%%%%%%%%%%%%%%%%%%%%%%%%%


\newcommand{\basedir}{fablab-document}
\documentclass{\basedir/fablab-document}

% \usepackage{fancybox} %ovale Boxen für Knöpfe - nicht mehr benötigt
\usepackage{amssymb} % Symbole für Knöpfe
\usepackage{subfigure,caption}
\usepackage{eurosym}
\usepackage{tabularx} % Tabellen mit bestimmtem Breitenverhältnis der Spalten
\usepackage{wrapfig} % Textumlauf um Bilder
\usepackage{tikz}

\renewcommand{\texteuro}{\euro}

\definecolor{dunkelgruen}{rgb}{0,0.4,0}

\newcommand{\img}[2]{
	\begin{figure}[ht]%
	\includegraphics[width=\columnwidth]{img/#1}%
	\caption{#2}%
	\end{figure}}
	
\newcommand\circlearound[1]{%
  \tikz[baseline]\node[draw,shape=circle,anchor=base] {#1} ;}
	
\newcounter{ueberpunkte}
\setcounter{ueberpunkte}{1}
\renewcommand{\labelenumi}{\circlearound{\theueberpunkte}\stepcounter{ueberpunkte}}

\tikzstyle{ablaufGross} = [rectangle, draw, thick, draw=black, fill=red!40,
    text width=\textwidth, text centered, rounded corners, minimum height=6em]
\newcommand{\ueb}[1]{\textbf{#1}\\}
		
\newcommand{\kastenRot}[2]{
\begin{tikzpicture}
	\node[ablaufGross]
		{\ueb{#1} 
		#2};
\end{tikzpicture}}

\linespread{1.2}

\date{2015}
\author{Michael Jäger}
\fancyfoot[C]{kontakt@fablab.fau.de}
\title{Produkte anlegen im ERP}

\begin{document}

\section{Anleitung}

\subsection*{Bestellung im ERP anlegen}
\kastenRot{Wichtig}{Pro Projekt / Person jeweils eine einzelne Bestellung bei der MEW aufgeben und im ERP einzeln anlegen!}


\img{11_Angebot_erstellen.pdf}{Angebot im ERP anlegen und speichern}

\begin{tabular}{>{\itshape}l@{\qquad}l}
\multicolumn{2}{l}{\color{red}\textbf{Müssen ausgefüllt werden}} \\
Geschäftsbedingungen 							& Kurze Beschreibung für was oder wen bestellt wird \\
\multicolumn{2}{l}{\color{orange}\textbf{Bei Bestellung von Auftragsarbeiten beim Schreiner}} \\
Falls Schubladen oder ähnliches 				& Artikel \textbf{[0441] Auftragsfertigung Schreiner} verwenden und \\
und es nicht verkauft wird						& mit den voraussichtlichen Kosten anlegen \\
\multicolumn{2}{l}{\color{dunkelgruen}\textbf{Für das MEW Bestellformular wichtig}} \\
PO Nummer 		& \\
Bestelldatum 	& \\
\end{tabular}


\newpage 
\subsection*{Bestellzettel der MEW ausfüllen}
\img{12_Materialbestellformular_mw.pdf}{Ausschnitt aus dem Bestellformular der MEW}

\begin{tabular}{>{\itshape}l@{\qquad}l}
\multicolumn{2}{l}{\color{red}\textbf{Pflichtfelder - Müssen wie im ERP ausgefüllt werden}} \\
Projekt/Kostenst.											& PO Nummer aus dem ERP \textbf{und} ein Wort als Beschreibung / Person \\
															& \quad \textcolor[rgb]{0.4,0.4,0.4}{z.B. PO00001 - Laser} \\
															& \quad \textcolor[rgb]{0.4,0.4,0.4}{z.B. PO00002 - Mustermann} \\
Datum														& Bestelldatum aus dem ERP \\[2ex]
\multicolumn{2}{l}{\color{orange}\textbf{Pflichtfelder}} \\
\multicolumn{2}{l}{\textbf{Nachbestellung Material (TG77)}} \\
Lehrstuhl 														& Lst. Informatik 3 \\
Lehrstuhlkürzel													& i3-FabLab \\
Kundennummer 													& 11031 \\
Kundenstatus													& normalerweise 1b (bei nicht wirtschaftlicher Tätigkeit), ansonst 1a \\
Name															& Name des Bestellers \\[1ex]
\multicolumn{2}{l}{\textbf{Bestellung auf Studienzuschüsse (TG96)}} \\
Lehrstuhl 														& Lst. Informatik 3 \\
Lehrstuhlkürzel													& StuZu \\
Kundennummer 													& 11032 \\
Kundenstatus													& 1b \\
Name															& Name des Bestellers \\[1ex]
\multicolumn{2}{l}{\textbf{Privatbestellungen von Leuten (TG77)}} \\
Lehrstuhl 														& Lst. Informatik 3 \\
Lehrstuhlkürzel													& Sonstiges \\
Kundennummer 													& 11033 \\
Kundenstatus													& normalerweise 1b (bei nicht wirtschaftlicher Tätigkeit), ansonst 1a \\
Name															& Name des Bestellers \\[2ex]
\multicolumn{2}{l}{\color{dunkelgruen}\textbf{Optional}} \\
Telefon																& Telefonnumer oder E-Mail Adresse, man bekommt bescheid wenn es fertig ist \\
\end{tabular}

\newpage 
\subsection*{MEW Rechnung verarbeiten}
to be continued

\end{document}
